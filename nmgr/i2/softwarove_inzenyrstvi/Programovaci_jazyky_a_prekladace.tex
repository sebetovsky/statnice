\newpage
\section{Programovací jazyky a překladače}
\begin{pozadavky}
\begin{pitemize}
\item Struktura kompilátoru a navazujících nástrojů (linkery, loadery, debuggery, knihovny, preprocesory).
\item Konečné automaty a lexikální analýza.
\item Syntaktická analýza - LL, LR techniky.
\item Syntaxí řízený překlad a atributové gramatiky.
\item Reprezentace programu mezikódem.
\item Překlad výrazů a programových struktur.
\item Rozsahy platnosti proměnných, aktivační záznamy, implementace vnořených procedur, volací konvence.
\item Vliv architektury počítače na generování kódu a optimalizaci.
\item Metody generování kódu, přidělování registrů, scheduling, optimalizace.
\item Podpora kompilátorů pro synchronizační primitiva, vlákna.
\item Objektově orientované jazyky a principy jejich implementace.
\item Překladače vs. interpretry, skriptovací jazyky.
\end{pitemize}
\end{pozadavky}

\subsection{Struktura kompilátoru a navazujících nástrojů (linkery, loadery, debuggery, knihovny, preprocesory).}
\subsection{Konečné automaty a lexikální analýza.}
\subsection{Syntaktická analýza - LL, LR techniky.}
\subsection{Syntaxí řízený překlad a atributové gramatiky.}
\subsection{Reprezentace programu mezikódem.}
\subsection{Překlad výrazů a programových struktur.}
\subsection{Rozsahy platnosti proměnných, aktivační záznamy, implementace vnořených procedur, volací konvence.}
\subsection{Vliv architektury počítače na generování kódu a optimalizaci.}
\subsection{Metody generování kódu, přidělování registrů, scheduling, optimalizace.}
\subsection{Podpora kompilátorů pro synchronizační primitiva, vlákna.}
\subsection{Objektově orientované jazyky a principy jejich implementace.}
\subsection{Překladače vs. interpretry, skriptovací jazyky.}
